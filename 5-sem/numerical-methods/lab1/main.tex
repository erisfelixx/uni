\documentclass[a4paper,12pt]{article}
\usepackage[utf8]{inputenc}
\usepackage[ukrainian]{babel}
\usepackage{geometry}
\usepackage{amsmath}
\usepackage{graphicx}
\usepackage{float}
\geometry{left=25mm,right=15mm,top=20mm,bottom=20mm}

\begin{document}

% --- титульна сторінка ---
\begin{titlepage}
    \centering
    {\large Київський національний університет імені Тараса Шевченка \par}
    {\large Факультет комп'ютерних наук та кібернетики \par}
    {\large Кафедра інтелектуальних програмних систем \par}
    {\large Чисельні методи в інформатиці \par}
    \vspace{5cm}

    {\LARGE \textbf{Звіт} \par}
    {\Large з лабораторної роботи №1 \par}
    {\Large «Розв’язування нелінійних рівнянь» \par}
    \vspace{1cm}

    {\Large Варіант №7 \par}
    \vspace{1cm}

    {\large студентки 3-го курсу \par}
    {\large групи ІПС-31 \par}
    {\Large \textbf{Совгирі Анни} \par}

    \vfill

    {\large Київ --- 2025}
\end{titlepage}

% --- Основна частина ---
\newpage

\section*{Вступ}
У даній роботі необхідно знайти найменший додатний корінь нелінійного
рівняння вигляду $f(x)=0$ методом простої ітерації та методом релаксації.
Процес розв’язання включає попереднє дослідження функції з метою визначення
інтервалів, на яких можливе існування коренів, подальше виділення областей,
що містять лише один корінь, а також вибір початкового наближення для
ітераційних процедур.

Після цього виконується обчислення кореня з наперед заданою точністю за
допомогою обраних алгоритмів. Крім того, проводиться як апріорна, так і
апостеріорна оцінка кількості ітерацій, що дозволяє оцінити ефективність
застосованих методів.

\section*{Умова задачі}
Знайти найменший додатний корінь нелінійного рівняння
\[
x^4 + x^3 - 6x^2 + 20x - 16 = 0
\]
методом простої ітерації та релаксації з точністю $\varepsilon = 10^{-4}$.
Знайти апріорну та апостеріорну оцінку кількості кроків. Початковий проміжок
та початкове наближення обрати однакове для обох методів (якщо це можливо),
порівняти результати роботи методів між собою.



\section*{Теорія}

\subsection*{Метод простої ітерації}
Метод простої ітерації ґрунтується на зведенні нелінійного рівняння
до вигляду
\[
    x = \varphi(x),
\]
де $\varphi(x) = x + \Psi(x)f(x)$, $\Psi(x)$ --- знакопостала неперервна функція.  

Початкове наближення обирається довільне з проміжку:
$x_0 \in [a;b]$, ітераційний процес має вигляд:
\begin{equation*}
    x_{n+1} = \varphi(x_n).
\end{equation*}

\textbf{Достатня умова збіжності.} Нехай для $\forall x_0 : x_0 \in S$,  
де $S = \{x : |x - x_0| \leq \delta\}$, $\varphi(x)$ задовольняє умовам:
\begin{enumerate}
    \item $\max\limits_{x \in S} |\varphi'(x)| \leq q < 1$;
    \item $|\varphi(x_0) - x_0| \leq (1-q)\delta$;
\end{enumerate}
тоді ітераційний процес збігається $\exists x^* : \lim\limits_{n \to \infty} x_n = x^*$,  
при чому швидкість збіжності лінійна:
\begin{equation*}
    |x_n - x^*| \leq \frac{q^n}{1-q}|\varphi(x_0) - x_0|.
\end{equation*}

\textit{Зауваження.} Замість умови 1) $\max\limits_{x \in S} |\varphi'(x)| \leq q < 1$  
можна використати умову Ліпшица: $|\varphi(x) - \varphi(y)| \leq q|x-y|$, $x,y \in S$.

З формули швидкості збіжності можна вивести апріорну оцінку кількості кроків:
\[
n \geq \left[ \frac{\ln \dfrac{|\varphi(x_0) - x_0|}{(1-q)\varepsilon}}{\ln(1/q)} \right] + 1.
\]


Умова припинення залежить від $q$:
\[
    |x_n - x_{n-1}| \leq \frac{1-q}{q}\varepsilon, \quad \text{якщо } q<\tfrac{1}{2};
\]
\[
    |x_n - x_{n-1}| \leq \varepsilon, \quad \text{в інших випадках}.
\]

\subsection*{Метод релаксації}
Якщо в методі простої ітерації $\Psi(x) \equiv \tau \equiv const$,  
то отримуємо метод релаксації.  

Початкове наближення обирається довільне з проміжку:
$x_0 \in [a;b]$, ітераційний процес:
\begin{equation}
    x_{n+1} = x_n \pm \tau f(x_n), \tag{18}
\end{equation}
де «$+$», якщо $f'(x)<0$; «$-$», якщо $f'(x)>0$.

\textbf{Достатня умова збіжності.} Якщо в ітераційному процесі параметр  
$\tau \in (0; 2/M_1)$, де $0 < m_1 < |f'(x)| < M_1$,  
$M_1 = \max\limits_{x \in [a;b]} |f'(x)|$, $m_1 = \min\limits_{x \in [a;b]} |f'(x)|$,  
то ітераційний процес збігається, при цьому швидкість збіжності лінійна.  

\textbf{Оптимальний параметр.} Якщо обрати
\[
    \tau_0 = \frac{2}{M_1+m_1},
\]
то кількість ітерацій буде мінімальною, швидкість збіжності залишається лінійною:
\[
    |x_n - x^*| \leq q_0^n |x_0 - x^*|, \quad q_0 = \frac{M_1 - m_1}{M_1 + m_1}.
\]

Для оптимального параметра $\tau_0$ апріорна оцінка кількості кроків:
\[
n_0 \geq \left[ \frac{\ln \dfrac{|x_0 - x^*|}{\varepsilon}}{\ln(1/q_0)} \right] + 1.
\]


Умова припинення ітераційного процесу:
\[
    |x_n - x_{n-1}| \leq \varepsilon.
\]



\section*{Розв’язання}

\subsection*{Графік функції}
\begin{figure}[H]
    \centering
    \includegraphics[width=0.75\textwidth]{pic1.png}
    \caption{Графік функції \( y = f(x) \)}
    \label{fig:function}
\end{figure}

\subsection*{Графік похідної функції}
\begin{figure}[H]
    \centering    \includegraphics[width=0.75\textwidth]{pic2.png}
    \caption{Графік похідної \( y = f'(x) \)}
    \label{fig:derivative}
\end{figure}



Ми бачимо, що $x=1$ є точним коренем рівняння. Щоб переконатися, що це найменший додатний корінь, проаналізуємо похідну: 
\[
f'(x)=4x^3+3x^2-12x+20
\]

На проміжку $[0,1]$:
\begin{itemize}
  \item $f'(0)=20>0$
  \item $f'(1)=4+3-12+20=15>0$
\end{itemize}

Оскільки похідна $f'(x)$ додатна на інтервалі $[0,1]$, функція $f(x)$ на цьому відрізку зростає.  
Оскільки $f(0)<0$ та $f(1)=0$, то на інтервалі $(0,1]$ існує лише один корінь, і це $x=1$.

\textbf {Для демонстрації роботи методів візьмемо ізольований інтервал навколо кореня, наприклад $[0.8,\,1.2]$.}

\subsection*{Метод простої ітерації}

\[
\varphi(x) = x + \Psi(x) f(x)
\]

Оберемо $\Psi(x) = -\frac{x}{20}$. Функція є лінійною, не константою, знакопостійною на відрізку [0.8; 1.2].  
Тоді:
\[
\varphi(x) = x + \Psi(x)f(x) = x - \frac{x}{20}(x^4 + x^3 - 6x^2 + 20x - 16)
\]
\hspace{1cm}

\textbf{Перевірка умов збіжності.}

\begin{figure}[h!]
    \centering
    \includegraphics[width=0.8\textwidth]{pic3.png}
    \caption{Графік похідної \( y = \varphi'(x) \)}
    \label{fig:phi_derivative}
\end{figure}


Метод збігається, якщо на обраному інтервалі виконується умова  
\[
|\varphi'(x)| \le q < 1.
\]

Знайдемо похідну:
\[
\varphi'(x) = 1 + \Psi'(x)f(x) + \Psi(x)f'(x)
\]
де  
\[
\Psi'(x) = -\frac{1}{20}, \quad
f(x) = x^4 + x^3 - 6x^2 + 20x - 16, \quad
f'(x) = 4x^3 + 3x^2 - 12x + 20.
\]
Підставимо:
\[
\varphi'(x) = 1 - \frac{1}{20}(x^4 + x^3 - 6x^2 + 20x - 16) - \frac{x}{20}(4x^3 + 3x^2 - 12x + 20).
\]

Перевіримо значення на межах інтервалу [0.8; 1.2]:
\[
|\varphi'(0.8)| = \left|\frac{20 - f(0.8) - 0.8 \cdot f'(0.8)}{20}\right|
= \left|\frac{20 - (-2.9184) - 0.8 \cdot 14.368}{20}\right|
= \frac{11.424}{20} \approx 0.5712
\]
\[
|\varphi'(1.2)| = \left|\frac{20 - f(1.2) - 1.2 \cdot f'(1.2)}{20}\right|
= \left|\frac{20 - 3.1616 - 1.2 \cdot 16.832}{20}\right|
= \frac{3.36}{20} \approx 0.1680
\]

Отже, максимальне значення на інтервалі:
\[
q = \max_{x \in [0.8;1.2]} |\varphi'(x)| \approx 0.5712.
\]
Оскільки $q = 0.5712 < 1$, умова збіжності виконується.

\[
|\varphi(x_0) - x_0| = |0.916736 - 0.8| = 0.11674
\]
\[
(1 - q)\delta = (1 - 0.5712)\cdot 0.4 = 0.17152
\]
Отже, і друга умова збіжності виконується.

\hspace{1cm}

\textbf{Кількість ітерацій $n$} для досягнення точності $\varepsilon$ можна оцінити за формулою:
\[
n \geq \left[ \frac{\ln \dfrac{|\varphi(x_0) - x_0|}{(1-q)\varepsilon}}{\ln(1/q)} \right] + 1
\]
Візьмемо початкове наближення $x_0 = 0.8$:
\[
\varphi(x_0) = \varphi(0.8) = 0.916736
\]

\[
n \geq \left[ \frac{\ln \dfrac{0.116736}{(1 - 0.5712) \cdot 10^{-4}}}{\ln(1 / 0.5712)} \right] + 1
= \left[ \frac{\ln(2722.29)}{\ln(1.75)} \right] + 1
= \left[ \frac{7.909}{0.560} \right] + 1
= [14.12] + 1 = 15
\]

Отже, очікувана кількість ітерацій $n \approx 15$.

\begin{table}[h!]
\centering
\caption{Таблиця ітерацій (метод простої ітерації)}
\begin{tabular}{|c|c|c|}
\hline
\textbf{n} & \textbf{$x_n$} & \textbf{$\Delta = |x_n - x_{n-1}|$} \\
\hline
0 & 0.8000000000 & -- \\
1 & 0.9167360000 & 0.11674 \\
2 & 0.9731610805 & 0.05643 \\
3 & 0.9926495534 & 0.01949 \\
4 & 0.9981139202 & 0.00546 \\
5 & 0.9995252812 & 0.00141 \\
6 & 0.9998811175 & 0.00036 \\
7 & \textbf {0.9999702667} & 0.00009 \\
8 & 0.9999925659 & 0.00002229 \\
9 & 0.9999981414 & 0.00000558 \\
10 & 0.9999995534 & 0.00000139 \\
11 & 0.9999998838 & 0.00000035 \\
12 & 0.9999999710 & 0.00000009 \\
13 & 0.9999999927 & 0.00000002 \\
14 & 0.9999999982 & 0.00000001 \\
15 & 0.9999999995 & 0.00000000 \\
\hline
\end{tabular}
\label{tab:iteration-simple}
\end{table}


Оскільки $q = 0.5712 > \tfrac{1}{2}$, то ми повинні використовувати другу умову припинення ітерацій:
\[
|x_n - x_{n-1}| \leq \varepsilon,
\]
Отже, зупинка на \textbf{7-й ітерації.}



\subsection*{Метод релаксації}

\[
m_1 = \min |f'(x)| = f'(0.8) = 4(0.8)^3 + 3(0.8)^2 - 12(0.8) + 20 = 14.368
\]

\[
M_1 = \max |f'(x)| = f'(1.2) = 16.832
\]

\textbf{Оптимальний параметр:}
\[
\tau_0 = \frac{2}{M_1 + m_1} = \frac{2}{16.832 + 14.368} = \frac{2}{31.2} \approx 0.0641
\]

\textbf{Швидкість збіжності:}
\[
q_0 = \frac{M_1 - m_1}{M_1 + m_1} = \frac{16.832 - 14.368}{16.832 + 14.368} = \frac{2.464}{31.2} \approx 0.0790
\]

\textbf{Апріорна оцінка кількості ітерацій:}
\[
n_0 \geq \left[ \frac{\ln \dfrac{|x_0 - x^*|}{\varepsilon}}{\ln(1/q_0)} \right] + 1
\]

\[
n_0 \geq \left[ \frac{\ln \left(\frac{0.4}{10^{-4}}\right)}{\ln(1/0.0790)} \right] + 1
= \left[ \frac{\ln(4000)}{\ln(12.66)} \right] + 1
= \left[ \frac{8.294}{2.538} \right] + 1
= [3.268] + 1 = 3 + 1 = 4
\]


Отже, очікувана кількість ітерацій \(n \approx 4\).

\begin{table}[h!]
\centering
\caption{Таблиця ітерацій (релаксація)}
\begin{tabular}{|c|c|c|}
\hline
$n$ & $x_n$ & $\Delta = |x_n - x_{n-1}|$ \\
\hline
0 & 0.8000000000 & -- \\
1 & 0.9870769231 & 0.18708 \\
2 & 0.9994715320 & 0.01239 \\
3 & 0.9999796206 & 0.00051 \\
4 & \textbf{0.9999992161} & 0.00002 \\
\hline
\end{tabular}
\end{table}

Умова припинення \(\,|x_n - x_{n-1}| \le \varepsilon\,\) виконується на \textbf{4-й ітерації}.

\section*{Висновок}

На основі проведеної лабораторної роботи, яка полягала у знаходженні найменшого 
додатного кореня нелінійного рівняння методами простої ітерації та релаксації, 
можна зробити наступний висновок. Обидва методи, застосовані на інтервалі 
ізоляції кореня \([0.8; 1.2]\), успішно зійшлися до точного кореня \(x^* = 1\).

Метод простої ітерації вимагав 7 ітерацій для досягнення заданої точності. 
Натомість, метод релаксації із застосуванням оптимального параметра продемонстрував 
значно вищу швидкість збіжності і досяг необхідної точності лише за 4 ітерації. 
Таким чином, незважаючи на необхідність додаткових обчислень для знаходження 
оптимального параметра на інтервалі, метод релаксації виявився більш ефективним 
з погляду мінімізації фактичної кількості ітераційних кроків порівняно з методом 
простої ітерації.


\end{document}