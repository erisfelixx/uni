\documentclass[a4paper,12pt]{article}
\usepackage[utf8]{inputenc}
\usepackage[ukrainian]{babel}
\usepackage{geometry}
\usepackage{amsmath}
\usepackage{graphicx}
\usepackage{float}
\geometry{left=25mm,right=15mm,top=20mm,bottom=20mm}

\begin{document}

% --- титульна сторінка ---
\begin{titlepage}
    \centering
    {\large Київський національний університет імені Тараса Шевченка \par}
    {\large Факультет комп'ютерних наук та кібернетики \par}
    {\large Кафедра інтелектуальних програмних систем \par}
    {\large Чисельні методи в інформатиці \par}
    \vspace{5cm}

    {\LARGE \textbf{Звіт} \par}
    {\Large з лабораторної роботи №2 \par}
    {\Large «Методи розв’язання систем лiнiйних алгебраїчних рiвнянь» \par}
    \vspace{1cm}

    {\Large Варіант №3 \par}
    \vspace{1cm}

    {\large студентки 3-го курсу \par}
    {\large групи ІПС-31 \par}
    {\Large \textbf{Совгирі Анни} \par}

    \vfill

    {\large Київ --- 2025}
\end{titlepage}

% --- Основна частина ---
\newpage

\section*{Вступ}
Метою роботи є програмна реалізація та дослідження методів розв'язання систем лінійних алгебраїчних рівнянь (СЛАР), а саме:
\begin{itemize}
    \item метод Гауса (г.е. по матриці);
    \item метод прогонки;
    \item метод Зейделя.
\end{itemize}

Для виконання лабораторної роботи необхідно згенерувати квадратну матрицю розміром $4 \times 4$ з цілими елементами за модулем менше $10$ та вектор правої частини, що задовольняють умови збіжності обраного методу. 
Також буде реалізовано алгоритм, що дозволяє знаходити розв'язок системи із заданою користувачем точністю~$\varepsilon$.



\section*{Теорія}

\subsection*{Метод Гаусса}

Метод Гаусса є прямим методом розв’язання систем лінійних алгебраїчних рівнянь
\[
Ax=b,\qquad A\in \mathbb{R}^{n\times n}, \qquad \det A\neq 0.
\]

Він складається з двох етапів: \textbf{прямий хід} (перетворення до трикутної матриці)
та \textbf{зворотний хід} (послідовне знаходження змінних).

\[
\begin{cases}
a_{11}x_1 + a_{12}x_2 + \ldots + a_{1n}x_n = a_{1(n+1)},\\
a_{21}x_1 + a_{22}x_2 + \ldots + a_{2n}x_n = a_{2(n+1)},\\
\qquad \vdots \\
a_{n1}x_1 + a_{n2}x_2 + \ldots + a_{nn}x_n = a_{n(n+1)}.
\end{cases}
\]
\[
a_{i(n+1)} = b_i,\qquad i=\overline{1,n}.
\]


\subsubsection*{Прямий хід}

\[
a^{(k)}_{kj} = \frac{a^{(k-1)}_{kj}}{a^{(k-1)}_{kk}}, \qquad k=\overline{1,n},
\]
\[
a^{(k)}_{ij} = a^{(k-1)}_{ij} - a^{(k-1)}_{ik}\, a^{(k)}_{kj},
\qquad
j=\overline{k+1,n+1},\; i=\overline{k+1,n},
\]
\[
a^{(k-1)}_{kk} \neq 0.
\]

Після першого кроку система набуває вигляду:
\[
\begin{cases}
x_1 + a^{(1)}_{12} x_2 + \ldots + a^{(1)}_{1n} x_n = a^{(1)}_{1(n+1)},\\
x_2 + \ldots + a^{(2)}_{2n} x_n = a^{(2)}_{2(n+1)},\\
\qquad \vdots \\
x_n = a^{(n)}_{n(n+1)}.
\end{cases}
\]

\subsubsection*{Зворотний хід}

\[
x_n = a^{(n)}_{n(n+1)},
\]
\[
x_i = a^{(i)}_{i(n+1)} - \sum_{j=i+1}^{n} a^{(i)}_{ij} x_j,
\qquad i=\overline{n-1,1}.
\]

---

\subsection*{Вибір головного елемента}

Для зменшення обчислювальної похибки в методі Гаусса використовують вибір головного елемента:

\begin{enumerate}
\item по стовпцях,
\item по рядках,
\item \textbf{за всією матрицею}.
\end{enumerate}

У випадку вибору головного елемента \textbf{за всією матрицею} на кожному кроці
вибирається елемент найбільшого модуля у всій поточній підматриці
та виконуються перестановки рядків і стовпців.


\subsection*{Метод прогонки}

Метод прогонки використовується, якщо матриця системи лінійних
алгебраїчних рівнянь $A$ є тридіагональною. Цей метод є частковим
випадком методу Гаусса.

Нехай маємо систему вигляду:
\[
\begin{cases}
-c_0 y_0 + b_0 y_1 = -f_0,\\[2mm]
\qquad \vdots \\[2mm]
a_i y_{i-1} - c_i y_i + b_i y_{i+1} = -f_i,\quad i = 1, n-1,\\[2mm]
\qquad \vdots \\[2mm]
a_n y_{n-1} - c_n y_n = -f_n;
\end{cases}
\]

\textbf{Достатня умова стійкості.}
Нехай коефіцієнти $a_0, b_0 = 0$; $c_0, c_n \ne 0$; $a_i, b_i, c_i \ne 0$;
$i = 1, n-1$. Якщо виконуються умови:
\[
\begin{aligned}
1)\;& |c_i| \ge |a_i| + |b_i|,\quad i = 0, n;\\
2)\;& \exists\, i:\; |c_i| > |a_i| + |b_i|,
\end{aligned}
\]
то метод є стійким: $|\alpha_i| \le 1$; $|z_i| > 1$, $i = 1,n$.

\medskip

Прямий хід методу Гаусса в методі прогонки відповідає знаходженню
прогонкових коефіцієнтів:
\[
\alpha_1 = \frac{b_0}{c_0}, \qquad
\beta_1 = \frac{f_0}{c_0},
\]
\[
\alpha_{i+1} = \frac{b_i}{z_i}, \qquad
\beta_{i+1} = \frac{f_i + a_i \beta_i}{z_i},
\]
\[
z_i = c_i - \alpha_i a_i,\qquad i = 1, n-1.
\]

\textbf{Зворотний хід:}
\[
y_n = \frac{f_n + a_n \beta_n}{z_n},\qquad
y_i = \alpha_{i+1}y_{i+1} + \beta_{i+1},\quad i = n-1, 0.
\]

Складність методу прогонки:
\[
Q(n) = 8n - 2.
\]


\subsection*{Метод Зейделя}

Метод Зейделя є ітераційним методом для розв’язання 
СЛАР $Ax = b$, т.т. розв’язок знаходимо із заданою точністю $\varepsilon$.
Початкове наближення $x^{0}$ обираємо довільним чином.
Ітераційний процес має вигляд:
\[
x^{k+1}_i = 
-\sum_{j=1}^{i-1} \frac{a_{ij}}{a_{ii}} x^{k+1}_j
-\sum_{j=i+1}^{n} \frac{a_{ij}}{a_{ii}} x^{k}_j
+ \frac{b_i}{a_{ii}}. \tag{22}
\]

\textbf{Достатня умова збіжності 1.}  
Якщо $\forall\, i : i = 1,n$ виконується нерівність
\[
|a_{ii}| \ge \sum_{\substack{j=1 \\ j \ne i}}^{n} |a_{ij}|,
\]
то ітераційний процес методу Зейделя (22) збігається, при чому
швидкість збіжності лінійна.

\textbf{Достатня умова збіжності 2.}  
Якщо $A = A^{T} > 0$, то ітераційний процес методу Зейделя (22) 
збігається, при чому швидкість збіжності лінійна.

Умова припинення:
\[
\| x^{n} - x^{n-1} \| \le \varepsilon.
\]


\vspace{1cm}
\subsubsection*{Тридіагональна матриця}

Тридіагональна матриця має ненульові елементи лише на головній діагоналі, а також на сусідніх діагоналях. Це особливий вид матриці, для якого підходить метод прогонки, який забезпечує ефективне розв’язання з меншими обчислювальними витратами.

\subsubsection*{Діагонально домінантна матриця}

Матриця є діагонально домінантною, якщо кожен діагональний елемент більший за суму модулів елементів у відповідному рядку (окрім діагонального елемента).



\section*{Розв’язання}
У ході роботи була згенерована тридіагональна та діагонально домінантна матриця розміром $4 \times 4$ із цілими елементами в межах від $-9$ до $9$, а також сформовано відповідний вектор правої частини. Для обчислення розв’язку отриманої системи було застосовано три чисельні методи: метод Гаусса з вибором головного елемента, метод прогонки та метод Зейделя.

\begin{figure}[h!]
    \centering
    \includegraphics[width=0.85\textwidth]{pic0.png}
    \caption{Початкові дані}
\end{figure}

\subsection*{Метод Гаусса}

Маємо систему
\[
A\mathbf{x} = \mathbf{b}, \qquad
A =
\begin{pmatrix}
5 & 3 & 0 & 0\\
2 & 8 & -3 & 0\\
0 & -3 & 7 & -3\\
0 & 0 & 1 & 4
\end{pmatrix},
\qquad
\mathbf{b} =
\begin{pmatrix}
-9\\ 3\\ 2\\ -8
\end{pmatrix}.
\]

Початкова розширена матриця:
\[
\left[
\begin{array}{cccc|c}
5 & 3 & 0 & 0 & -9\\
2 & 8 & -3 & 0 & 3\\
0 & -3 & 7 & -3 & 2\\
0 & 0 & 1 & 4 & -8
\end{array}
\right].
\]

%------------------------------------------------------------
\paragraph{Крок 1. Вибір головного елемента.}
Найбільший за модулем елемент у всій матриці – це \(8\) (елемент \(a_{22}\)).
Виконуємо перестановку рядків \(R_1 \leftrightarrow R_2\) та стовпців
\(C_1 \leftrightarrow C_2\) (повний вибір головного елемента):
\[
\left[
\begin{array}{cccc|c}
8 & 2 & -3 & 0 & 3\\
3 & 5 &  0 & 0 & -9\\
-3& 0 &  7 & -3& 2\\
0 & 0 &  1 & 4 & -8
\end{array}
\right].
\]

Виконуємо виключення елементів у першому стовпці під головним:
\[
m_{21} = \frac{3}{8}, \qquad
m_{31} = -\frac{3}{8}, \qquad
m_{41} = 0.
\]
Отримуємо
\[
\left[
\begin{array}{cccc|c}
8 & 2 & -3 & 0 & 3\\[2pt]
0 & \dfrac{17}{4} & \dfrac{9}{8} & 0 & -\dfrac{81}{8}\\[6pt]
0 & \dfrac{3}{4}  & \dfrac{47}{8} & -3 & \dfrac{25}{8}\\[6pt]
0 & 0 & 1 & 4 & -8
\end{array}
\right].
\]

%------------------------------------------------------------
\paragraph{Крок 2. Вибір у підматриці.}
У підматриці рядків та стовпців максимальний за
модулем елемент – \(\dfrac{47}{8}\) (у позиції \((3,3)\)).
Переставляємо рядки \(R_2 \leftrightarrow R_3\) та стовпці
\(C_2 \leftrightarrow C_3\):
\[
\left[
\begin{array}{cccc|c}
8 & -3 & 2 & 0 & 3\\[2pt]
0 & \dfrac{47}{8} & \dfrac{3}{4} & -3 & \dfrac{25}{8}\\[6pt]
0 & \dfrac{9}{8}  & \dfrac{17}{4} & 0 & -\dfrac{81}{8}\\[6pt]
0 & 1 & 0 & 4 & -8
\end{array}
\right].
\]

Виконуємо виключення елементів у другому стовпці:
\[
m_{32} = \frac{\frac{9}{8}}{\frac{47}{8}} = \frac{9}{47}, \qquad
m_{42} = \frac{1}{\frac{47}{8}} = \frac{8}{47}.
\]
Після цього:
\[
\left[
\begin{array}{cccc|c}
8 & -3 & 0 & 2 & 3\\[2pt]
0 & \dfrac{47}{8} & -3 & \dfrac{3}{4} & \dfrac{25}{8}\\[6pt]
0 & 0 & \dfrac{193}{47} & \dfrac{27}{47} & -\dfrac{504}{47}\\[6pt]
0 & 0 & -\dfrac{6}{47} & \dfrac{212}{47} & -\dfrac{401}{47}
\end{array}
\right].
\]

%------------------------------------------------------------
\paragraph{Крок 3. Вибір у останній підматриці.}
У підматриці найбільший за модулем
елемент – \(\dfrac{212}{47}\) (позиція \((4,4)\)).
Переставляємо рядки \(R_3 \leftrightarrow R_4\) та стовпці
\(C_3 \leftrightarrow C_4\):
\[
\left[
\begin{array}{cccc|c}
8 & -3 & 0 & 2 & 3\\[2pt]
0 & \dfrac{47}{8} & -3 & \dfrac{3}{4} & \dfrac{25}{8}\\[6pt]
0 & 0 & \dfrac{212}{47} & -\dfrac{6}{47} & -\dfrac{401}{47}\\[6pt]
0 & 0 & \dfrac{27}{47} & \dfrac{193}{47} & -\dfrac{504}{47}
\end{array}
\right].
\]

Виконуємо виключення елемента в позиції \((4,3)\):
\[
m_{43} = \frac{\frac{27}{47}}{\frac{212}{47}} = \frac{27}{212}.
\]
Отримаємо остаточну верхньотрикутну систему
\[
\left[
\begin{array}{cccc|c}
8 & -3 & 0 & 2 & 3\\[2pt]
0 & \dfrac{47}{8} & -3 & \dfrac{3}{4} & \dfrac{25}{8}\\[6pt]
0 & 0 & \dfrac{212}{47} & -\dfrac{6}{47} & -\dfrac{401}{47}\\[6pt]
0 & 0 & 0 & \dfrac{437}{106} & -\dfrac{2043}{212}
\end{array}
\right].
\]

Через перестановки стовпців порядок невідомих у трикутній системі
відповідає змінним \((x_2, x_3, x_4, x_1)\), але після зворотного ходу
ми повертаємося до початкового порядку \(x_1,x_2,x_3,x_4\).

\paragraph{Зворотний хід.}
Із останнього рівняння:
\[
x_1 = \frac{-\frac{2043}{212}}{\frac{437}{106}}
    = -\frac{2043}{874}.
\]
Далі послідовно знаходимо \(x_4, x_3, x_2\) (підставляючи вже відомі
значення) і в результаті одержуємо
\[
x_1 = -\frac{2043}{874},\qquad
x_2 = \frac{783}{874},\qquad
x_3 = -\frac{74}{437},\qquad
x_4 = -\frac{1711}{874}.
\]

Отже, розв'язок системи в початковому порядку невідомих має вигляд
\[
\mathbf{x} =
\begin{pmatrix}
-\dfrac{2043}{874}\\[4pt]
\dfrac{783}{874}\\[4pt]
-\dfrac{74}{437}\\[4pt]
-\dfrac{1711}{874}
\end{pmatrix}
=
\begin{pmatrix}
-2.338\\[2pt]
0.895\\[2pt]
-0.169\\[2pt]
-1.957
\end{pmatrix}.
\]

\begin{figure}[h!]
    \centering
    \includegraphics[width=0.60\textwidth]{pic1.png}
    \caption{Результати програмної реалізації (Гаус)}
\end{figure}


\subsection*{Метод прогонки}

Матриця є тридіагональною, отже можемо використати метод прогонки. 
Маємо:
$y_i = x_{i+1}$ і
\[
\begin{aligned}
&c_0 = -5,\quad b_0 = 3,\\
&a_1 = 2,\quad c_1 = -8,\quad b_1 = -3,\\
&a_2 = -3,\quad c_2 = -7,\quad b_2 = -3,\\
&a_3 = 1,\quad c_3 = -4.
\end{aligned}
\]

\subsubsection*{Прямий хід}

За формулами з теорії
\[
\alpha_1 = \frac{b_0}{c_0},\quad
\beta_1 = \frac{f_0}{c_0},\quad
\alpha_{i+1} = \frac{b_i}{z_i},\quad
\beta_{i+1} = \frac{f_i + a_i \beta_i}{z_i},\quad
z_i = c_i - \alpha_i a_i,
\]
одержуємо:

\[
\begin{aligned}
\alpha_1 &= \frac{3}{-5} = -\frac{3}{5}, &
\beta_1 &= \frac{9}{-5} = -\frac{9}{5},\\[4pt]
z_1 &= c_1 - \alpha_1 a_1
    = -8 - \Bigl(-\frac{3}{5}\cdot 2\Bigr)
    = -\frac{34}{5},\\[6pt]
\alpha_2 &= \frac{b_1}{z_1}
         = \frac{-3}{-\frac{34}{5}}
         = \frac{15}{34}, &
\beta_2 &= \frac{f_1 + a_1 \beta_1}{z_1}
        = \frac{3 + 2\cdot\Bigl(-\frac{9}{5}\Bigr)}{-\frac{34}{5}}
        = \frac{33}{34},\\[6pt]
z_2 &= c_2 - \alpha_2 a_2
    = -7 - \frac{15}{34}\cdot(-3)
    = -\frac{193}{34},\\[6pt]
\alpha_3 &= \frac{b_2}{z_2}
         = \frac{-3}{-\frac{193}{34}}
         = \frac{102}{193}, &
\beta_3 &= \frac{f_2 + a_2 \beta_2}{z_2}
        = \frac{2 + (-3)\cdot\frac{33}{34}}{-\frac{193}{34}}
        = \frac{167}{193},\\[6pt]
z_3 &= c_3 - \alpha_3 a_3
    = -4 - \frac{102}{193}
    = -\frac{874}{193}.
\end{aligned}
\]

\subsubsection*{Зворотний хід}

З останнього рівняння:
\[
y_3 = \frac{f_3 + a_3 \beta_3}{z_3}
    = \frac{8 + 1\cdot\frac{167}{193}}{-\frac{874}{193}}
    = -\frac{1711}{874}.
\]

Далі послідовно:
\[
\begin{aligned}
y_2 &= \alpha_3 y_3 + \beta_3
    = \frac{102}{193}y_3 + \frac{167}{193}
    = -\frac{74}{437},\\[4pt]
y_1 &= \alpha_2 y_2 + \beta_2
    = \frac{15}{34}y_2 + \frac{33}{34}
    = \frac{783}{874},\\[4pt]
y_0 &= \alpha_1 y_1 + \beta_1
    = -\frac{3}{5}y_1 - \frac{9}{5}
    = -\frac{2043}{874}.
\end{aligned}
\]

Оскільки $y_i = x_{i+1}$, остаточний розв'язок у початковому порядку невідомих:
\[
\mathbf{x} =
\begin{pmatrix}
x_1\\
x_2\\
x_3\\
x_4
\end{pmatrix}
=
\begin{pmatrix}
y_0\\
y_1\\
y_2\\
y_3
\end{pmatrix}
=
\begin{pmatrix}
-\dfrac{2043}{874}\\[4pt]
\dfrac{783}{874}\\[4pt]
-\dfrac{74}{437}\\[4pt]
-\dfrac{1711}{874}
\end{pmatrix}
\approx
\begin{pmatrix}
-2.338\\[2pt]
0.896\\[2pt]
-0.169\\[2pt]
-1.957
\end{pmatrix}.
\]
\begin{figure}[h!]
    \centering
    \includegraphics[width=0.60\textwidth]{pic2.png}
    \caption{Результати програмної реалізації (прогонка)}
\end{figure}


\subsection*{Метод Зейделя}

Маємо діагонально домінантну систему:
\[
\begin{pmatrix}
5 & 3 & 0 & 0\\
2 & 8 & -3 & 0\\
0 & -3 & 7 & -3\\
0 & 0 & 1 & 4
\end{pmatrix}
\begin{pmatrix}
x_1\\ x_2\\ x_3\\ x_4
\end{pmatrix}
=
\begin{pmatrix}
-9\\ 3\\ 2\\ -8
\end{pmatrix}.
\]

Перепишемо її в еквівалентному вигляді, виражаючи кожну змінну через інші:
\[
\begin{cases}
x_1 = \dfrac{-9 - 3x_2}{5},\\[4pt]
x_2 = \dfrac{3 - 2x_1 + 3x_3}{8},\\[4pt]
x_3 = \dfrac{2 + 3x_2 + 3x_4}{7},\\[4pt]
x_4 = \dfrac{-8 - x_3}{4}.
\end{cases}
\]

Ітераційний процес методу Зейделя (з використанням “нових” значень по мірі їх появи) має вигляд:
\[
\begin{cases}
x_1^{(k+1)} = \dfrac{-9 - 3x_2^{(k)}}{5},\\[6pt]
x_2^{(k+1)} = \dfrac{3 - 2x_1^{(k+1)} + 3x_3^{(k)}}{8},\\[6pt]
x_3^{(k+1)} = \dfrac{2 + 3x_2^{(k+1)} + 3x_4^{(k)}}{7},\\[6pt]
x_4^{(k+1)} = \dfrac{-8 - x_3^{(k+1)}}{4}.
\end{cases}
\]

Початкове наближення вибираємо
\[
x^{(0)} =
\begin{pmatrix}
0\\ 0\\ 0\\ 0
\end{pmatrix}.
\]

Перші ітерації:
\[
x^{(1)} =
\begin{pmatrix}
-1.8000\\
0.8250\\
0.6393\\
-2.1598
\end{pmatrix},\qquad
x^{(2)} =
\begin{pmatrix}
-2.2950\\
1.1885\\
-0.1306\\
-1.9674
\end{pmatrix}.
\]

Подальші обчислення виконуються програмно до виконання умови зупинки
\[
\|x^{(k+1)} - x^{(k)}\|_\infty \le \varepsilon,
\]
де в роботі взято $\varepsilon = 10^{-6}$.

\begin{figure}[h!]
    \centering
    \includegraphics[width=0.90\textwidth]{pic5.png}
    \caption{Ітераційні кроки}
\end{figure}


Збіжність досягається за 13 ітерацій, остаточне наближення розв'язку має вигляд
\[
x^{(*)} \approx
\begin{pmatrix}
-2.3375\\
0.8959\\
-0.1693\\
-1.9577
\end{pmatrix}.
\]

\begin{figure}[h!]
    \centering
    \includegraphics[width=0.60\textwidth]{pic3.png}
    \caption{Результати програмної реалізації (Зейдель)}
\end{figure}

\section*{Висновок}

\begin{enumerate}
    \item \textbf{Метод Гаусса з вибором головного елемента.}  
    Це універсальний прямий метод, який можна застосовувати до будь-якої квадратної матриці. Він забезпечує високу точність і стійкість, однак є ресурсоємним для великих систем через значні обчислювальні витрати та можливе накопичення похибок округлення. Метод Гаусса доцільно використовувати для систем середнього розміру або коли матриця не має спеціальної структури.

    \item \textbf{Метод прогонки (Томаса).}  
    Є найефективнішим методом для систем із тридіагональною структурою матриці. Забезпечує високу точність і має мінімальну обчислювальну складність. Обмеженням є те, що метод застосовується лише до строго діагонально-домінантних тридіагональних матриць і не може бути використаний для загальних систем. 
    \item \textbf{Метод Зейделя.}  
    Ітераційний метод, який вирізняється простотою реалізації та ефективністю для великих або розріджених систем. Його збіжність залежить від вибору початкового наближення та діагональної домінантності матриці. За слабкої або відсутньої домінантності може збігатися повільно або не збігтися. Водночас для добре структурованих систем він дає ефективний керований наближений результат.
\end{enumerate}


\end{document}