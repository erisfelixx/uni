\documentclass[a4paper,12pt]{article}
\usepackage[utf8]{inputenc}
\usepackage[ukrainian]{babel}
\usepackage{geometry}
\usepackage{amsmath}
\usepackage{graphicx}
\usepackage{float}
\geometry{left=25mm,right=15mm,top=20mm,bottom=20mm}

\begin{document}

% --- титульна сторінка ---
\begin{titlepage}
    \centering
    {\large Київський національний університет імені Тараса Шевченка \par}
    {\large Факультет комп'ютерних наук та кібернетики \par}
    {\large Кафедра інтелектуальних програмних систем \par}
    {\large Чисельні методи в інформатиці \par}
    \vspace{5cm}

    {\LARGE \textbf{Звіт} \par}
    {\Large з лабораторної роботи №4 \par}
    {\Large «Методи інтерполяції функцій: поліном Ньютона та обернена інтерполяція» \par}
    \vspace{1cm}

    {\Large Варіант №1 \par}
    \vspace{1cm}

    {\large студентки 3-го курсу \par}
    {\large групи ІПС-31 \par}
    {\Large \textbf{Совгирі Анни} \par}

    \vfill

    {\large Київ --- 2025}
\end{titlepage}

% --- Основна частина ---
\newpage
\section*{Вступ}

Метою лабораторної роботи є дослідження інтерполяційних алгоритмів, аналіз точності наближення та формування практичних навичок програмної реалізації чисельних методів. У рамках лабораторної роботи задається аналітична функція $\boldsymbol{\tg(x)}$ на відрізку $\boldsymbol{[-0.5;\ 0.5]}$, за якою формується таблиця з не менше ніж \textbf{15 вузлів}. На основі цих даних будується інтерполяційний поліном Ньютона, проводиться обчислювальний експеримент та порівняння отриманих результатів з аналітичним значенням функції.  

Також розв’язується задача оберненої інтерполяції: для вибраного значення $y^\*$, що лежить усередині області значень функції і не входить до таблиці, визначається відповідне значення аргументу $x$. Усі етапи роботи супроводжуються програмною реалізацією, чисельними обчисленнями та побудовою графічних матеріалів.



\section*{Теорія}

\subsection*{Інтерполяційний поліном Ньютона}

\textit{Розділеною різницею першого порядку} називається величина:
\[
f(x_i, x_j) = \frac{f(x_j) - f(x_i)}{x_j - x_i};
\]

\textit{другого порядку:}
\[
f(x_{i-1}, x_i, x_{i+1}) = \frac{f(x_i, x_{i+1}) - f(x_{i-1}, x_i)}{x_{i+1} - x_{i-1}};
\]

\textit{(k+1) порядку:}
\[
f(x_i, \ldots, x_{i+k+1}) = \frac{f(x_{i+1}, \ldots, x_{i+k+1}) - f(x_i, \ldots, x_{i+k})}{x_{i+k+1} - x_i}.
\]

\textbf{Таблиця розділених різниць} має вигляд:
\[
\begin{array}{c|c c c c c}
x_0 & f(x_0) & f(x_0;x_1) & f(x_0;x_1;x_2) & \cdots & f(x_0;x_1;\ldots;x_n) \\
x_1 & f(x_1) & f(x_1;x_2) & f(x_1;x_2;x_3) & \cdots & \\
x_2 & f(x_2) & f(x_2;x_3) & & & \\
\vdots & \vdots & \vdots & & \ddots & \\
x_n & f(x_n) & & & & 
\end{array}
\]

На підставі цієї таблиці, використовуючи перший її рядок, можемо записати
\textit{інтерполянт Ньютона вперед:}
\[
P_n(x) = f(x_0) + f(x_0, x_1)(x - x_0) + f(x_0, x_1, x_2)(x - x_0)(x - x_1) + \ldots
\]
\[
+ \ f(x_0, x_1, \ldots, x_n)(x - x_0)(x - x_1)\cdots(x - x_{n-1}),
\]

чи скориставшись останнім рядком, дістанемо
\textit{інтерполяційну формулу Ньютона назад:}
\[
P_n(x) = f(x_n) + f(x_{n-1};x_n)(x - x_n) + \ldots
\]
\[
\ldots + f(x_0;x_1;\ldots;x_n)(x - x_n)(x - x_{n-1})\cdots(x - x_1).
\]

\textit{Зауваження.} За $(n+1)$ вузлом можна побудувати інтерполяційний поліном не вище $n$-го степеня, тобто степінь може бути нижчим. Для визначення степеня інтерполяційного поліному зручно використовувати розділені різниці.

\par\vspace{2mm}
\textbf{Похибка інтерполяції:} для оцінки похибки інтерполяції можна використати оцінку залишкового члена у формі Ньютона:
\[
|f(x) - L_n(x)| = \omega(x) f(x, x_0, x_1, \ldots, x_n).
\]



\subsection*{Обернена інтерполяція}

Нехай функція $y = f(x) \in C[a,b]$, що задана таблично $(x_i, y_i),\ i=\overline{0,n}$, монотонна. Для знаходження $x^*$ застосовуємо такий алгоритм:

\par\vspace{2mm}
Будуємо за таблицею $(x_i, y_i),\ i=\overline{0,n}$ таку таблицю: $(y_i, x_i),\ i=\overline{0,n}$. На підставі останньої таблиці інтерполянт набуває вигляду:
\[
L_n(y)=\sum_{i=0}^{n} x_i \frac{\omega_{n+1}(y)}{(y-y_i)\,\omega_{n+1}'(y_i)},
\]
де $\omega_{n+1}(y) = (y-y_0)(y-y_1)\cdots(y-y_n)$ та $L(y^*) \approx x^*$.

\par\vspace{2mm}
Залишковий член в цьому випадку утворюється із залишкового члена формули Ньютона, якщо в останньому поміняти місцями $x$ та $y$, а похідну $f'(x)$ замінити на похідну від оберненої функції. \textbf{Похибка інтерполяції} має вигляд:
\[
|x - x^*| \leq \frac{\widetilde{M}_{n+1}}{(n+1)!}\, |\omega(y^*)|,
\qquad 
\widetilde{M}_{n+1} = \max_y \left| \frac{d^{n+1}}{dy^{n+1}} x(y) \right|.
\]

\section*{Розв’язання}

У межах цієї лабораторної роботи розглядається функція
\[
f(x) = \tg(x),
\]
визначена на відрізку $[-0.5;\, 0.5]$. Даний проміжок повністю належить області визначення тангенса, оскільки його межі значно віддалені від точок розриву $\frac{\pi}{2} + k\pi$. Завдяки цьому функція на вибраному інтервалі є неперервною.

\begin{figure}[h!]
    \centering
    \includegraphics[width=0.5\textwidth]{pic1.png}
    \caption{Графік функції y=tg(x)}
    \label{fig:my_image}
\end{figure}

Оскільки похідна тангенса має вигляд
\[
f'(x)=\sec^2(x)=\frac{1}{\cos^2(x)}>0,
\]
то на відрізку $[-0.5;\, 0.5]$ вона є додатною для всіх значень $x$. Це означає, що функція $\tg(x)$ є \textbf{строго монотонно зростаючою} на всій області дослідження. Монотонність забезпечує коректність подальшої постановки задачі оберненої інтерполяції та дозволяє однозначно відновлювати значення аргументу за відомими значеннями функції.

\begin{figure}[h!]
    \centering
    \includegraphics[width=0.5\textwidth]{pic2.png}
    \caption{$f'(x)=\sec^{2}(x)=\dfrac{1}{\cos^{2}(x)}$}
    \label{fig:my_image}
\end{figure}

Значення функції на межах відрізку становлять:
\[
\tg(-0.5)\approx -0.5463,\qquad \tg(0.5)\approx 0.5463,
\]
тобто область значень охоплює приблизно $[-0.55;\, 0.55]$. Саме всередині цього діапазону можна коректно вибирати значення для задачі оберненої інтерполяції.

Для подальшого чисельного дослідження формується таблиця з 15 вузлів на відрізку $[-0.5;\,0.5]$. Вузли обираються рівновіддалено, що дозволяє рівномірно охопити весь проміжок та отримати достатню густоту точок для побудови інтерполяційного полінома. 






\subsection{Інтерполяція Ньютона}

Для наближення функції $\tg(x)$ на відрізку $[-0.5;\,0.5]$ використано інтерполяційний поліном Ньютона на розділених різницях. Спочатку було обрано $15$ рівновіддалених вузлів
\[
x_i = -0.5 + i\cdot h,\quad i=\overline{0,14}, \qquad 
h = \frac{0.5-(-0.5)}{14} = \frac{1}{14} \approx 0.071428,
\]
та обчислено відповідні значення $y_i = \tg(x_i)$. На основі цієї таблиці побудовано таблицю розділених різниць $f[x_i;\dots;x_j]$ до $14$-го порядку. 

Перший рядок таблиці розділених різниць містить коефіцієнти інтерполяційного полінома Ньютона. Позначимо
\[
c_0 = f[x_0],\quad
c_1 = f[x_0,x_1],\quad
c_2 = f[x_0,x_1,x_2],\ \ldots,\ 
c_{14} = f[x_0,x_1,\ldots,x_{14}].
\]
Тоді інтерполяційний поліном має вигляд
\[
P_{14}(x) = 
c_0 
+ c_1(x - x_0)
+ c_2(x - x_0)(x - x_1)
+ \dots
+ c_{14}(x - x_0)(x - x_1)\cdots(x - x_{13}),
\]
де числові значення коефіцієнтів $c_k$ отримані безпосередньо з першого рядка таблиці розділених різниць.

\begin{figure}[h!]
    \centering
    \includegraphics[width=1.0\textwidth]{pic3.png}
    \caption{Програмна реалізація таблиці розділених різниць}
    \label{fig:my_image}
\end{figure}
\begin{figure}[h!]
    \centering
    \includegraphics[width=1.0\textwidth]{pic4.png}
    \caption{Поліном у формі Ньютона та спрощений}
    \label{fig:my_image}
\end{figure}


На основі отриманого полінома обчислювались наближені значення $P_{14}(x)$ у контрольних точках всередині відрізка, які порівнювалися з точними значеннями $\tg(x)$. Абсолютні похибки виявилися малими (порядку $10^{-n}$ для обраних точок), що підтверджує коректність побудови таблиці розділених різниць та реалізації методу Ньютона. На основі обчислених значень було побудовано графік інтерполяційного полінома та порівняно його з графіком вихідної функції $\tg(x)$ на відрізку $[-0.5;\,0.5]$.


\begin{figure}[h!]
    \centering
    \includegraphics[width=0.8\textwidth]{pic5.png}
    \caption{Порівняння графіків}
    \label{fig:my_image}
\end{figure}



\subsection{Обернена інтерполяція}

Оскільки на відрізку $[-0.5;\,0.5]$ функція $\tg(x)$ є строго монотонно зростаючою, задача оберненої інтерполяції коректно сформульована: для будь-якого значення $y^\*$ з області значень функції існує єдиний відповідний аргумент.

Для реалізації методу оберненої інтерполяції початкову табличну функцію $(x_i, y_i)$ було перетворено у вигляд $(y_i, x_i)$. 

Далі для послідовності $(y_i, x_i)$ було побудовано таблицю розділених різниць за тим самим алгоритмом, що й у прямій інтерполяції Ньютона. Перший рядок цієї таблиці містить коефіцієнти інтерполяційного полінома для функції $x = Q(y)$, який має вигляд
\[
Q_{14}(y) = 
d_0 
+ d_1(y - y_0)
+ d_2(y - y_0)(y - y_1)
+ \dots
+ d_{14}(y - y_0)(y - y_1)\cdots(y - y_{13}),
\]
де $d_k = f[y_0, y_1, \dots, y_k]$ — відповідні розділені різниці з першого рядка побудованої таблиці.

\begin{figure}[h!]
    \centering
    \includegraphics[width=1.0\textwidth]{pic6.png}
    \caption{Таблиця розділених різниць для оберненої інтерполяції}
    \label{fig:my_image}
\end{figure}
\begin{figure}[h!]
    \centering
    \includegraphics[width=1.0\textwidth]{pic7.png}
    \caption{Поліном для оберненої інтерполяції}
    \label{fig:my_image}
\end{figure}

Отриманий поліном $Q(y)$ є наближенням до оберненої функції $\arctan(y)$, оскільки саме вона є аналітичним розв’язком рівняння $\tg(x)=y$. Для оцінки точності було обрано значення $y^\*$ з внутрішності області значень функції та обчислено наближене значення, після чого його порівняно з точним значенням.

Отримана абсолютна похибка виявилася дуже малою, що підтверджує коректність побудованого полінома $Q(y)$ та правильність реалізації оберненої інтерполяції.Також, інтерполяційна крива практично збігається з графіком оберненої функції на всьому досліджуваному проміжку.

\begin{figure}[h!]
    \centering
    \includegraphics[width=0.8\textwidth]{pic8.png}
    \caption{Поліном для оберненої інтерполяції}
    \label{fig:my_image}
\end{figure}



\section*{Висновок}

У роботі було досліджено застосування інтерполяційних методів до функції $\tg(x)$ на відрізку $[-0.5; 0.5]$. Побудований інтерполяційний поліном Ньютона продемонстрував високу точність наближення: числові похибки в контрольних точках виявилися невеликими, а графічне порівняння показало практичний збіг інтерполяційного полінома з вихідною функцією. Монотонність тангенса на цьому проміжку також дала змогу коректно виконати обернену інтерполяцію, отримавши поліном, що наближує функцію $\arctan(y)$. Отримані результати підтверджують ефективність методу розділених різниць та його здатність забезпечувати якісне чисельне наближення як функції, так і її оберненої.

\end{document}
