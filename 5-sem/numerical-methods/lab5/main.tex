\documentclass[a4paper,12pt]{article}
\usepackage[utf8]{inputenc}
\usepackage[ukrainian]{babel}
\usepackage{geometry}
\usepackage{amsmath}
\usepackage{graphicx}
\usepackage{float}
\geometry{left=25mm,right=15mm,top=20mm,bottom=20mm}

\begin{document}

% --- титульна сторінка ---
\begin{titlepage}
    \centering
    {\large Київський національний університет імені Тараса Шевченка \par}
    {\large Факультет комп'ютерних наук та кібернетики \par}
    {\large Кафедра інтелектуальних програмних систем \par}
    {\large Чисельні методи в інформатиці \par}
    \vspace{5cm}

    {\LARGE \textbf{Звіт} \par}
    {\Large з лабораторної роботи №5 \par}
    {\Large «Сплайни» \par}
    \vspace{1cm}

    {\Large Варіант №7 \par}
    \vspace{1cm}

    {\large студентки 3-го курсу \par}
    {\large групи ІПС-31 \par}
    {\Large \textbf{Совгирі Анни} \par}

    \vfill

    {\large Київ --- 2025}
\end{titlepage}

% --- Основна частина ---
\newpage
\section*{Вступ}
У даній лабораторній роботі розглядаються чисельні методи інтерполяції та, зокрема, побудова природного кубічного інтерполяційного сплайна. Сплайни є одним із найпоширеніших інструментів для гладкого наближення функцій, оскільки забезпечують неперервність самої функції, а також її перших і других похідних у всіх вузлах інтерполяції. 

Метою роботи є побудова природного кубічного інтерполяційного сплайна для функції
\[
f(x) = \tan(x)
\]
на відрізку \( x \in [-0.5,\, 0.5] \) за тими самими вузлами, що були використані в лабораторній роботі №4. 

У межах роботи необхідно:
\begin{itemize}
    \item сформувати систему лінійних алгебраїчних рівнянь для визначення коефіцієнтів \( c_i = S''(x_i) \) згідно з формулами методички;
    \item розв'язати тридіагональну систему методом прогонки;
    \item отримати коефіцієнти кубічних поліномів на кожному відрізку;
    \item побудувати графіки функції, сплайна та їхніх похідних;
    \item провести чисельне порівняння точного значення функції з інтерполяційним наближенням.
\end{itemize}


\section*{Теорія}
\subsection*{Інтерполяційний природний кубічний сплайн}

Інтерполяційним природнім кубічним сплайном називається поліном, для якого виконуються умови:
\begin{enumerate}
    \item $s(x)$ --- поліном степеня $3$ для $x \in [x_{i-1}, x_i],\; i = \overline{1,n};$
    \item $s(x) \in C^2_{[a;b]};$
    \item $s(x_i) = f(x_i),\; i = \overline{0,n};$
    \item $s''(a) = s''(b) = 0$ --- умова природності.
\end{enumerate}

\noindent
\textbf{Зауваження.} Для побудови інтерполяційного кубічного сплайну можна замість умови 4) використовувати інші умови, але тоді сплайн не буде природнім: $s''(a)=A;\; s''(b)=B$ або $s'(a)=A;\; s'(b)=B$, або умови періодичності: $s(a)=s(b)$, $s'(a)=s'(b)$, $s''(a)=s''(b)$.

\bigskip

Розглянемо формули для побудови інтерполяційного природного кубічного сплайну $s_i$ на проміжку $[x_{i-1}, x_i]$:
\[
s_i = a_i + b_i(x - x_i) + \frac{c_i}{2}(x - x_i)^2 + \frac{d_i}{6}(x - x_i)^3,
\]
де $c_i$ знаходяться з тридіагональної системи лінійних алгебраїчних рівнянь:
\[
h_i c_{i-1} + 2c_i(h_i + h_{i+1}) + h_{i+1}c_{i+1}
=
6\left( \frac{f_{i+1} - f_i}{h_{i+1}} - \frac{f_i - f_{i-1}}{h_i} \right),
\]
\[
c_0 = c_n = 0;
\]

\noindent
решта коефіцієнтів знаходяться за формулами:
\[
a_i = f_i,\qquad
b_i = \frac{h_i}{2}c_i - \frac{h_i^2}{6}d_i + \frac{f_i - f_{i-1}}{h_i},\qquad
d_i = \frac{c_i - c_{i-1}}{h_i}.
\]


\section*{Розв’язання}

Для побудови природного кубічного інтерполяційного сплайна на відрізку 
\([a,b] = [-0.5,\, 0.5]\) використаємо \(n+1 = 15\) вузлів, рівномірно розподілених на даному проміжку.
Крок сітки дорівнює
\[
h = \frac{b-a}{n} = \frac{0.5 - (-0.5)}{14} = \frac{1}{14} \approx 0.0714286.
\]
Отже, значення вузлів \(x_i\) та відповідні значення функції \(f(x)=\tan(x)\) мають вигляд:
\[
\begin{array}{c|c|c}
i & x_i & f_i = \tan(x_i) \\ \hline
0 & -0.500000 & -0.546302 \\
1 & -0.428571 & -0.456893 \\
2 & -0.357143 & -0.373144 \\
3 & -0.285714 & -0.293751 \\
4 & -0.214286 & -0.217622 \\
5 & -0.142857 & -0.143837 \\
6 & -0.071429 & -0.071550 \\
7 &  0.000000 &  0.000000 \\
8 &  0.071429 &  0.071550 \\
9 &  0.142857 &  0.143837 \\
10&  0.214286 &  0.217622 \\
11&  0.285714 &  0.293751 \\
12&  0.357143 &  0.373144 \\
13&  0.428571 &  0.456893 \\
14&  0.500000 &  0.546302
\end{array}
\]

Усі кроки \(h_i = x_{i+1}-x_i\) є однаковими:
\[
h_i = x_{i+1} - x_i = 0.0714286,\qquad i = 0,\dots, 13.
\]
















\subsection*{Побудова системи лінійних рівнянь для коефіцієнтів \(c_i\)}

Згідно з теорією, на кожному внутрішньому вузлі \(x_i,\; i = 1,\dots,n-1\) виконується рівняння
\[
h_i c_{i-1} + 2(h_i + h_{i+1})c_i + h_{i+1}c_{i+1}
=
6\left(
\frac{f_{i+1} - f_i}{h_{i+1}}
-
\frac{f_i - f_{i-1}}{h_i}
\right),
\]
а для природного сплайна додатково маємо граничні умови
\[
c_0 = 0,\qquad c_n = 0.
\]

У нашому випадку сітка рівномірна, тому \(h_i = h_{i+1} = h\), і коефіцієнти при невідомих набувають вигляду
\[
h\,c_{i-1} + 4h\,c_i + h\,c_{i+1}
=
6\left(
\frac{f_{i+1} - f_i}{h}
-
\frac{f_i - f_{i-1}}{h}
\right),
\qquad i = 1,\dots,13.
\]
Таким чином, будується тридіагональна система лінійних алгебраїчних рівнянь
\[
A \mathbf{c}_{\text{in}} = \mathbf{b},
\]
де \(\mathbf{c}_{\text{in}} = (c_1,\dots,c_{13})^T\) — вектор внутрішніх коефіцієнтів, а матриця \(A\) та вектор \(\mathbf{b}\) формуються напряму з наведених вище формул.

\begin{figure}[h!]
    \centering
    \includegraphics[width=1.0\textwidth]{pic1.png}
    \caption{Система лінійних алгебраїчних рівнянь}
    \label{fig:my_image}
\end{figure}




\subsection*{Розв’язання СЛАР методом прогонки}

Оскільки матриця \(A\) є тридіагональною, для розв’язання системи використовується \textbf{метод прогонки}. На прямому ході обчислюються допоміжні коефіцієнти \(\alpha_i\) та \(\beta_i\), після чого на зворотному ході послідовно визначаються усі внутрішні значення
\[
c_1, c_2, \dots, c_{13}.
\]
Граничні значення відновлюються згідно з умовою природності:
\[
c_0 = 0,\qquad c_{14} = 0.
\]

У результаті розв’язання СЛАР отримуємо повний набір коефіцієнтів
\[
\{c_i\}_{i=0}^{14},
\]
які інтерпретуються як значення другої похідної сплайна в вузлах: \(c_i = S''(x_i)\).

\begin{figure}[h!]
    \centering
    \includegraphics[width=0.3\textwidth]{pic2.png}
    \caption{\(c_i = S''(x_i)\)}
    \label{fig:my_image}
\end{figure}


\subsection*{Обчислення коефіцієнтів поліномів \(a_i, b_i, c_i, d_i\)}

На кожному відрізку \([x_i, x_{i+1}]\) сплайн подається кубічним поліномом
\[
s_i(x) = a_i + b_i(x - x_i) + \frac{c_i}{2}(x - x_i)^2 + \frac{d_i}{6}(x - x_i)^3.
\]
Коефіцієнти \(a_i, b_i, d_i\) обчислюються за формулами з методички:
\[
a_i = f_i,\qquad
b_i = \frac{f_{i+1} - f_i}{h_i} - \frac{h_i}{6}\left(2c_i + c_{i+1}\right),\qquad
d_i = \frac{c_{i+1} - c_i}{h_i},
\]
де \(h_i = x_{i+1} - x_i\). У програмній реалізації ці формули використовуються без додаткових перетворень: у циклі по \(i = 0,\dots,13\) для кожного відрізка обчислюються відповідні \(a_i, b_i\) та \(d_i\), що повністю визначає сплайн на всьому проміжку \([-0.5, 0.5]\).

\[
\begin{array}{c|r|r|r}
\hline
i & a_i & b_i & d_i \\ \hline
0  & -0.546302 &  1.269325 & -3.448436 \\
1  & -0.456893 &  1.216543 &  1.709264 \\
2  & -0.373144 &  1.137141 &  0.192374 \\
3  & -0.293751 &  1.086846 &  0.503941 \\
4  & -0.217627 &  1.047208 &  0.355599 \\
5  & -0.143837 &  1.020727 &  0.353925 \\
6  & -0.071550 &  1.005106 &  0.333332 \\
7  &  0.000000 &  1.000000 &  0.333332 \\
8  &  0.071550 &  1.005106 &  0.353925 \\
9  &  0.143837 &  1.020727 &  0.355599 \\
10 &  0.217627 &  1.047208 &  0.503941 \\
11 &  0.293751 &  1.086846 &  0.192374 \\
12 &  0.373144 &  1.137141 &  1.709264 \\
13 &  0.456893 &  1.216543 & -3.448436 \\
\hline
\end{array}
\]




\subsection*{Побудова сплайну}

\begin{figure}[h!]
    \centering
    \includegraphics[width=1.0\textwidth]{pic3.png}
    \caption{Побудова сплайну}
    \label{fig:my_image}
\end{figure}
\begin{figure}[h!]
    \centering
    \includegraphics[width=0.6\textwidth]{picplus.png}
    \caption{Сплайн у зведеному вигляді}
    \label{fig:my_image}
\end{figure}




\subsection*{Графіки та аналіз результатів}

Для аналізу якості інтерполяції було побудовано:
\begin{itemize}
    \item графік вихідної функції $f(x)=\tan(x)$ та сплайна $S(x)$;
    \item графік першої похідної $f'(x)$ та похідної сплайна $S'(x)$;
    \item графік другої похідної $f''(x)$ та другої похідної сплайна $S''(x)$;
    \item графіки абсолютних похибок: $|f-S|$, $|f'-S'|$, $|f''-S''|$.
\end{itemize}


\begin{figure}[h!]
    \centering
    \includegraphics[width=1.0\textwidth]{pic4.png}
    \caption{Графіки вихідної функції, першої та другої похідної}
    \label{fig:my_image}
\end{figure}


Графіки показують, що на всьому проміжку $[-0.5,0.5]$ сплайн дуже точно наближує функцію $\tan(x)$: криві $f(x)$ та $S(x)$ практично збігаються. Перша похідна $S'(x)$ також добре відтворює поведінку $f'(x)$, що відповідає гладкості класу $C^2$, притаманній кубічним сплайнам.

Для другої похідної $S''(x)$ видно характерні «злами» у вузлах, що природно, оскільки $S''(x)$ є кусочно-лінійною функцією. Незважаючи на це, наближення другої похідної є достатньо точним усередині інтервалів.

Аналіз похибок показує:
\begin{itemize}
    \item похибка $|f - S|$ на всьому проміжку мала (практично нульова), що свідчить про високу точність інтерполяції;
    \item похибка першої похідної також незначна та рівномірно мала;
    \item похибка другої похідної зростає на краях проміжку, що є типовою особливістю природних сплайнів, оскільки на кінцях накладається умова $S''(a)=S''(b)=0$.
\end{itemize}


\section*{Висновок}

У роботі було побудовано природний кубічний інтерполяційний сплайн для функції 
$f(x)=\tan(x)$ на відрізку $[-0.5,\,0.5]$ за заданою системою вузлів. 
Розв’язавши тридіагональну систему лінійних рівнянь, отримано значення другої похідної сплайна $c_i=S''(x_i)$, що дозволило однозначно визначити всі коефіцієнти кубічних поліномів на кожному підвідрізку.

Графічний аналіз показав, що сплайн дуже точно відтворює функцію $\tan(x)$ та її першу похідну на всьому проміжку. Похибка апроксимації є малою та рівномірною всередині інтервалу, а характерні зростання похибки другої похідної на кінцях відрізку пояснюються умовою природності $S''(a)=S''(b)=0$.

Отже, природний кубічний сплайн продемонстрував високу точність і стабільність, підтвердивши ефективність сплайнової інтерполяції для наближення функції та її похідних.




\end{document}
